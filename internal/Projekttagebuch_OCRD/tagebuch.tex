\begin{NoHyper}

\begin{LARGE}
\textbf{Planung:} OCRD Modulprojekt 3:
\glqq Automatische Nachkorrektur historischer OCR-erfasster Drucke 
mit integrierter optionaler interaktiver Nachkorrektur\grqq{} 
\end{LARGE}

\tableofcontents
\end{NoHyper}

\section{Stand der Implementierungen}

\section{Meetings}
\subsection {Kick-Off-Treffen 05.3 - 06.3 (Wolfenbüttel)}

\begin{itemize}
  \item \href{https://wiki.de.dariah.eu/pages/viewpage.action?pageId=64949522}{Offizielles Protokoll}
  \item Schwerpunkt: Gemeinsame Datenformate (METS + PAGE) für Eingabe und Ausgabe.
\end{itemize}

\subsection {Videokonferenz  20.4}

\begin{itemize}
\item \href{https://wiki.de.dariah.eu/display/OCR/2018-04-20+Besprechungsnotizen}{Offizielles Protokoll
}
\item Stand des Modulprojekts erläutert.
\item Vorstellung der OCRD-CLI Schnittstelle.
\end{itemize}


\subsection {1. Entwicklertreffen 26.6 - 27.6 (Berlin)}
\begin{itemize}
  \item \href{https://wiki.de.dariah.eu/display/OCR/1.+Entwickler-Workshop}{Offizielles Protokoll}
  \item Schwerpunkt: Integration des Modulprojekts in OCRD Workflow / Workspace-Umgebung.
  \item Diskussion mit Leipzig:  
  \begin{itemize}
    \item Wie können Alignierungsergebnisse gemeinsam genutzt werden ?
      \begin{itemize}
    \item PAGE.xml als Austauschformat für Alignierungen multipler OCR mit GT.
    \item \texttt{Textequiv} zur Speicherung alternativer Alignierungen?
    \end{itemize}
	\item Gemeinsame Nutzung des neuronalen Sprachemodell
    \item Verständigung auf baldiges Treffen in München zur weiteren gemeinsamen Abstimmung.
  \end{itemize}
  \item Gespräch mit Kay:
  \begin{itemize}
    \item Frage nach der Erstellung von (multipler) OCR zu Testzwecken.
    \item Zeilenalignierung wird nicht benötigt, man kann davon ausgehen, dass Zeilen korrekt
    zueinander aligniert sind.
  \end{itemize}
  
\end{itemize}

\subsection {Treffen mit Robert und Lena 07.8 - 08.8 (München)}
\begin{itemize}
\item \href{https://git.informatik.uni-leipzig.de/ls36hiqo/ocr-d/wikis/planung#7-8-august-2018-am-cis-in-m%C3%BCnchen-mit-florian-und-tobias}{Protokoll}

\item Absprachen zum Format des PAGE.XMLs für Alignierung.
\item Gemeinsame Nutzung PAGE-Adapters.
\item Gemeinsame Nutzung der RNN Sprachmodelle.
\end{itemize}
\subsection {Videokonferenz  18.9}
\begin{itemize}
  \item \href{https://wiki.de.dariah.eu/display/OCR/2018-09-18+Besprechungsnotizen}{Offizielles Protokoll}
  \item Vorstellung Stand + Rückmeldung über positive Erfahrungen mit OCRD-Core Anbindung.
  \item Ansprechen der Problematik fehlender OCR-Beispieldaten für die Nachkorrektur.
  
\end{itemize}


